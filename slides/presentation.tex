\documentclass{beamer}
\usepackage{fontspec} 
\usepackage{listings}
\usetheme{Warsaw}
\title[Scala solution with a Haskell taste]{Scala solution with a Haskell taste\\How to make your scala more haskellish?}
\author{George Leontiev}
\institute{folone.info}
\date{February 27, 2013}

\begin{document}

\begin{frame}
\titlepage
\end{frame}


\begin{frame}{Introduction}
  Note: I intentionally made it more "interesting" to show more neat scalaz stuff
\end{frame}

\begin{frame}[fragile]
\frametitle{Brief Overview}
Let's look at the types\newline
\lstset{escapechar=\@}
Main functions
\begin{lstlisting}
  object Words
    wordCount :: String → [(String, Int)]
    main :: Array String → @$\rlap{----}{IO}$@ Unit
\end{lstlisting}
Helper functions
\begin{lstlisting}
  acceptedChars :: Char → Boolean
  getFileContent :: String → IO String
  time :: R → IO R
\end{lstlisting}
\end{frame}

\begin{frame}[fragile]
\frametitle{Counting routine}
\begin{lstlisting}
def wordCount(text: String): List[(String, Int)] =
  text.filter(acceptedChars)
      .toLowerCase.split("\\W")
      .par
      .groupBy(identity)
      .map { case(key, value) =>
        key.trim → value.length
      }
      .filterNot { case(key, _) => key.isEmpty}
      .toList.sortBy { case(_, value) => -value }
      .seq.toList
\end{lstlisting}
\end{frame}

\begin{frame}[fragile]
\frametitle{First attempt}
\begin{lstlisting}
  def wholeFile(path: String) =
    for {
      source ← IO { Source.fromFile(path) }
      text   = source.mkString
      _      ← IO { source.close() }
      result = wordCount(text)
    } yield result.take(N).shows
\end{lstlisting}
\end{frame}

\begin{frame}{First attempt}
  Works fine, but eats all the heap on a large enouth file.
\end{frame}

\begin{frame}[fragile]
\frametitle{Second attempt}
\begin{lstlisting}
  def byLine(path: String) =
    for {
      source ← IO { Source.fromFile(path) }
      stream = source.getLines.toStream
      result = stream.map(wordCount)
        .foldLeft(Nil: List[(String, Int)]) {
          case(acc, v) =>
            acc |+| v
          }.sortBy { case(_, value) ⇒ -value }
      _      ← IO { source.close() }
    } yield result.take(N).shows
\end{lstlisting}
\end{frame}

\begin{frame}{Second attempt}
  Just what is this |+|?
\end{frame}

\begin{frame}[fragile]
\frametitle{Typeclasses}
\begin{lstlisting}
  implicit val mapInstances =
    new Show[List[(String, Int)]]
    with Monoid[List[(String, Int)]]

    is the same as

  instance Show [(String, Int)] where ...
  instance Monoid [(String, Int)] where ...
\end{lstlisting}
\end{frame}

\begin{frame}{Second attempt}
  Long story short:\newline
  Pretty slow, almost 2 minutes on mobi-dic.txt.
\end{frame}

\begin{frame}{What do we do?}
  Is idiomatic code doomed to be slow or blow up the heap?
\end{frame}

\begin{frame}[fragile]
\frametitle{Iteratees to the resque}
\end{frame}

\begin{frame}{Wordcounting}
\begin{enumerate}
  \item[Scoobi] http://nicta.github.com/scoobi/
  \item[Spark] http://spark-project.org/
  \item[Scalding] https://github.com/twitter/scalding/wiki/Type-safe-api-reference
\end{enumerate}
\end{frame}

\begin{frame}{Wordcounting}
  Turns out, this code will work for these "as is".
\end{frame}

\begin{frame}[fragile]
\frametitle{IO}
TODO finish up: some notes on IO and on Arrows
\end{frame}


\end{document}
